% \iffalse meta-comment
%
% Copyright (C) 2011 by Leo Liu <leoliu.pku@gmail.com>
% ---------------------------------------------------------------------------
% This work may be distributed and/or modified under the
% conditions of the LaTeX Project Public License, either version 1.3
% of this license or (at your option) any later version.
% The latest version of this license is in
%   http://www.latex-project.org/lppl.txt
% and version 1.3 or later is part of all distributions of LaTeX
% version 2005/12/01 or later.
%
% This work has the LPPL maintenance status `maintained'.
%
% The Current Maintainer of this work is Leo Liu.
%
% This work consists of the files diagbox.dtx and diagbox.ins
% and the derived filebase diagbox.sty.
%
% \fi
%
% \iffalse
%<*driver>
\ProvidesFile{diagbox.dtx}
%</driver>
%<package>\NeedsTeXFormat{LaTeX2e}[1999/12/01]
%<package>\ProvidesPackage{diagbox}
%<*package>
    [2011/11/23 v2.0 Making table heads with diagonal lines]
%</package>
%
%<*driver>
\documentclass[letterpaper]{ltxdoc}
\AtBeginDocument{
  \DeleteShortVerb{\|}
  \MakeShortVerb{\"}}
\usepackage{diagbox}
\usepackage[left=1.7in,top=1in,bottom=1in]{geometry}
\usepackage{fvrb-ex}
\usepackage[UTF8]{ctex}
\makeatletter
\renewcommand\glossary@prologue{%
  \pdfbookmark[1]{版本历史}{history}
  \section*{版本历史}
  \markboth{版本历史}{版本历史}}
\renewcommand\index@prologue{%
  \pdfbookmark[1]{Index / 代码索引}{index}
  \section*{Index / 代码索引}
  \markboth{Index / 代码索引}{Index / 代码索引}
  斜体的数字表示对应项说明所在的页码。下划线的数字表示定义所在的代码行号;而直
  立体的数字表示对应项使用时所在的行号。}
\makeatother
\usepackage[pdfstartview=FitH,bookmarksnumbered,colorlinks]{hyperref}
\EnableCrossrefs
\CodelineIndex
\RecordChanges
\begin{document}
  \DocInput{diagbox.dtx}
  \PrintChanges
  \PrintIndex
\end{document}
%</driver>
% \fi
%
% \CheckSum{521}
%
% \CharacterTable
%  {Upper-case    \A\B\C\D\E\F\G\H\I\J\K\L\M\N\O\P\Q\R\S\T\U\V\W\X\Y\Z
%   Lower-case    \a\b\c\d\e\f\g\h\i\j\k\l\m\n\o\p\q\r\s\t\u\v\w\x\y\z
%   Digits        \0\1\2\3\4\5\6\7\8\9
%   Exclamation   \!     Double quote  \"     Hash (number) \#
%   Dollar        \$     Percent       \%     Ampersand     \&
%   Acute accent  \'     Left paren    \(     Right paren   \)
%   Asterisk      \*     Plus          \+     Comma         \,
%   Minus         \-     Point         \.     Solidus       \/
%   Colon         \:     Semicolon     \;     Less than     \<
%   Equals        \=     Greater than  \>     Question mark \?
%   Commercial at \@     Left bracket  \[     Backslash     \\
%   Right bracket \]     Circumflex    \^     Underscore    \_
%   Grave accent  \`     Left brace    \{     Vertical bar  \|
%   Right brace   \}     Tilde         \~}
%
%
% \changes{v1.0}{2011/11/21}{初始版本。}
%
% \DoNotIndex{\newcommand, \newenvironment, \@nameuse, \@pkgextension,
% \@reserveda, \@tfor, \begin, \begingroup, \box, \csname, \def, \define@key,
% \dimexpr, \do, \dp, \edef, \else, \end, \endcsname, \endgroup, \endinput,
% \expandafter, \fi, \hbox, \ht, \ifcsname, \ifdim, \ifx, \Line, \makebox,
% \newbox, \newdimen, \p@, \PackageError, \pkg, \put, \relax, \RequirePackage,
% \setbox, \setkeys, \setlength, \strip@pt, \unexpanded, \unitlength,
% \vcenter, \wd, \xdef, \z@, \@ifnextchar, \bgroup, \FPadd, \FPmul, \FPneg,
% \FPsub, \FPupn, \let, \unless, \hspace}
%
% \providecommand*{\pkg}{\textsf}
% \GetFileInfo{diagbox.dtx}
% \title{\hypertarget{English}{\pkg{diagbox} Package (\fileversion)}
%   \makebox[0pt][l]{\hspace{3cm}\large
%     \hyperlink{Chinese}{$\Rightarrow$ \textsf{中文版}}}\\
%    Making Table Heads with Diagonal Lines}
% \author{Leo Liu \\ \path{leoliu.pku@gmail.com}}
% \date{\filedate}
%
% \maketitle
%
% \section{Introduction}
%
% \pkg{diagbox} is a replacement of old \pkg{slashbox} package\footnote{By
% Koichi Yasuoka and Toru Sato. Available on
% \url{CTAN://macros/latex/contrib/slashbox/slashbox.sty}}. I write this
% package simply because that \pkg{slashbox} is not available in \TeX\ Live
% for licening problems. \pkg{slashbox} has no explicit license information
% available, but \pkg{diagbox} is under LPPL.
%
% \pkg{diagbox} is a modern alternative of \pkg{slashbox}. I changed the
% user interface to use a key-value syntax, get rid of some restrictions of
% \pkg{slashbox}, use \pkg{pict2e} to draw diagonal lines. Especially, this
% package also provides ability to make a box with two diagonal lines in it.
% All these can be obtained by a "\diagbox" command.
%
% As a replacement of \pkg{slashbox}, \pkg{diagbox} package also provides
% compatible macros of \pkg{slashbox}, but the result is a little different.
%
% To use \pkg{diagbox}, \eTeX{} is needed. And \pkg{diagbox} requires
% \pkg{pict2e}, \pkg{keyval} and \pkg{fp} packages.
%
%
% \section{Usage}
%
% \subsection{Basic usage}
%
% To load the package, just put this in the preamble:
% \begin{Verbatim}
% \usepackage{diagbox}
% \end{Verbatim}
%
% \DescribeMacro{\diagbox}
% "\diagbox" is the main command. It can take two arguments, to produce a box
% with a diagonal line from north west to south east.
%
% For example:\\
% \begin{SideBySideExample}[frame=single,numbers=left,xrightmargin=.45\linewidth]
% \begin{tabular}{|l|ccc|}
% \hline
% \diagbox{Time}{Day} & Mon & Tue & Wed \\
% \hline
% Morning   & used & used &      \\
% Afternoon &      & used & used \\
% \hline
% \end{tabular}
% \end{SideBySideExample}
% \medskip
%
% "\diagbox" can also take three arguments, to draw a table head with two
% diagbox lines. For example,
% \begin{Example}[frame=single,numbers=left]
% \begin{tabular}{|l|ccc|}
% \hline
% \diagbox{Time}{Room}{Day} & Mon & Tue & Wed \\
% \hline
% Morning   & used & used &      \\
% Afternoon &      & used & used \\
% \hline
% \end{tabular}
% \end{Example}
%
% \subsection{More options}
%
% "\diagbox" can take a key-value list as an optional argument to specify the
% width and height of the box, the direction of the diagonal line, and the
% trimming margins:
% \begin{description}
% \item[width] Specify the width of the box explicitly. If it is omitted,
% package will calculate a width automatically.
% 
% \item[height] Specify the height of the box explicitly. If it is omitted,
% package will calculate a height automatically.
% 
% \item[dir] Specify the direction of the diagonal line. The value can be
% "NW", "NE", "SW" and "SE". Default value is "NW". The meaning of the values
% see below.
% \begin{itemize}
% \item "\diagbox[dir="\meta{direction}"]{A}{B}" in a table looks:\\
% \begin{tabular}{ll@{\qquad}ll}
% "NW" or "SE":&
% \begin{tabular}{|c|} \hline \diagbox[dir=SE]{A}{B} \\\hline \end{tabular} &
% "SW" or "NE":&
% \begin{tabular}{|c|} \hline \diagbox[dir=NE]{A}{B} \\\hline \end{tabular}
% \end{tabular}
% \item "\diagbox[dir="\meta{direction}"]{A}{M}{B}" in a table looks:\\
% \begin{tabular}{*4{ll@{\qquad}}}
% "NW"&
% \begin{tabular}{|c|} \hline \diagbox[dir=NW]{A}{M}{B} \\\hline \end{tabular}&
% "NE"&
% \begin{tabular}{|c|} \hline \diagbox[dir=NE]{A}{M}{B} \\\hline \end{tabular}&
% "SW"&
% \begin{tabular}{|c|} \hline \diagbox[dir=SW]{A}{M}{B} \\\hline \end{tabular}&
% "SE"&
% \begin{tabular}{|c|} \hline \diagbox[dir=SE]{A}{M}{B} \\\hline \end{tabular}
% \end{tabular}
% \end{itemize}
%
% \item[trim] Specify the margin to be trimmed. The value can be "l", "r", and
% "lr", "rl". This helps the slash line exceeds the boundary when "@{}" column
% specifier is used.
% \end{description}
%
% Here is a more complex example to show the usage of the options:
% \begin{Example}[frame=single,numbers=left]
% \begin{tabular}{|@{}l|c|c|r@{}|}
% \hline
% \diagbox[width=5em,trim=l]{Time}{Day} & Mon & Tue & Wed\\
% \hline
% Morning   & used & used & used\\
% \hline
% Afternoon &      & used & \diagbox[dir=SW,height=2em,trim=r]{A}{B} \\
% \hline
% \end{tabular}
% \end{Example}
%
% \bigskip
% What's more, you can use "\\" to break lines in "\diagbox". Manual setting
% of the height of the head may be needed. For example,
% \begin{Example}[frame=single,numbers=left]
% \begin{tabular}{|c|}
% \hline
% \diagbox[height=45pt]{line\\heads}{column\\heads} \\
% \hline
% \end{tabular}
% \end{Example}
%
% \subsection{Compatibility with \pkg{slashbox}}
%
% \pkg{diagbox} package emulates \pkg{slashbox} and also prevents
% \pkg{slashbox} to be loaded.
%
% \pkg{diagbox} package provides "\slashbox" and "\backslashbox" which syntax
% similar to \pkg{slashbox} package. However, the results of the two packages
% are a little different. These two commands are for compatibility only, it is
% better to use "\diagbox" instead for new documents.
%
% \DescribeMacro{\backslashbox}
% "\backslashbox" works as "\diagbox", but it takes two optional arguments to
% specify the "width" and "trim" options.
%
% \DescribeMacro{\slashbox}
% "\slashbox" works as "\diagbox[dir=SW]", and takes two optional arguments to
% specify the "width" and "trim" options.
%
% For example,\\
% \begin{SideBySideExample}[frame=single,numbers=left,xrightmargin=.45\linewidth]
% \begin{tabular}{|c|c|c|} \hline
% \backslashbox[2cm]{num}{alpha}
%   & A  & B  \\\hline
% 1 & A1 & B1 \\\hline
% 2 & A2 & B2 \\\hline
% \end{tabular}
% \end{SideBySideExample}
%
%
% \section{Known issues and TODO}
%
% Known issues:
% \begin{itemize}
% \item The result of "\slashbox" and "\backslashbox" is different with
% \pkg{slashbox} package. The algorithms to calculate the width and height are
% different; and the results of the second optional argument of "\slashbox"
% (i.e. "trim" key in "\diagbox") in the two packages are differernt.
%
% This is not a bug. Usually the width calculated by \pkg{diagbox} is more
% safe than \pkg{slashbox}.
%
% \item
% The cell with "\diagbox" should be the widest one of the column. Otherwise
% the slash line cannot exceeds the boundary. For example,\\[1ex]
% \begin{SideBySideExample}[frame=single,numbers=left,xrightmargin=.45\linewidth]
% \begin{tabular}{|c|}  \hline
%   \diagbox{A}{B}    \\\hline
%   Very long term    \\\hline
% \end{tabular}
% \end{SideBySideExample}
% \\[1ex]
% This can be solved by setting a wider "width" option of "\diagbox" manually.
% \end{itemize}
%
% TODO:
% \begin{itemize}
% \item
% Improve the document of the source code. The algorithm of "\diagbox@triple"
% should be explained in detail. However, the explanations would be only
% available in Chinese, I'm sorry.
% \end{itemize}
%
%
% \title{\hypertarget{Chinese}{\pkg{diagbox} 宏包(\fileversion)}
%   \makebox[0pt][l]{\hspace{3cm}\large
%     \hyperlink{English}{$\Rightarrow$ \textsf{English Version}}}\\
%    制做斜线表头}
% \author{刘海洋 \\ \path{leoliu.pku@gmail.com}}
% \date{\filedate}
%
% \maketitle
%
%
% \section{简介}
%
% \pkg{diagbox} 设计用来代替旧的 \pkg{slashbox} 宏包\footnote{作者 Koichi
% Yasuoka(安岡孝一)与 Sato Toru(佐藤徹)。宏包见
% \url{CTAN://macros/latex/contrib/slashbox/slashbox.sty}。}。编写这个宏包的缘
% 起是 \pkg{slashbox} 因为缺少明确的自由许可信息,被 \TeX\ Live 排除。这个宏包
% 是在 LPPL 协议下发行的。
%
% \pkg{diagbox} 是 \pkg{slashbox} 宏包的一个现代的版本。它采用了新的 key-value
% 式语法参数,去除了 \pkg{slashbox} 原有的一些长度限制,并调用 \pkg{pict2e} 宏
% 包画斜线;特别还添加了绘制两条斜线的表头的新功能。
%
% 作为 \pkg{slashbox} 的代替,\pkg{diagbox} 除了提供自己的新命令,也提供了
% \pkg{slashbox} 原有的两个命令,语法不变,编译结果略有区别。
%
% \pkg{diagbox} 依赖 \eTeX{} 扩展(这在目前总是可用的),依赖 \pkg{pict2e},
% \pkg{keyval} 和 \pkg{fp} 宏包。
%
% \section{用法说明}
%
% \subsection{基本用法}
%
% 要使用本宏包,首先在导言区调用:
% \begin{Verbatim}
% \usepackage{diagbox}
% \end{Verbatim}
%
% \DescribeMacro{\diagbox}
% "\diagbox" 是宏包提供的主要命令。它可以带有两个必选参数,表示要生成斜线表头的
% 两部分内容。默认斜线是从西北到东南方向的。
%
% 例如:\\
% \begin{SideBySideExample}[frame=single,numbers=left,xrightmargin=.45\linewidth]
% \begin{tabular}{|l|ccc|}
% \hline
% \diagbox{Time}{Day} & Mon & Tue & Wed \\
% \hline
% Morning   & used & used &      \\
% Afternoon &      & used & used \\
% \hline
% \end{tabular}
% \end{SideBySideExample}
% \medskip
%
% \changes{v2.0}{2011/11/22}{增加有三部分、双斜线的表头格式。}
% "\diagbox" 也可以接受三个参数,这样就会生成带有两条斜线的表头,例如:
% \begin{Example}[frame=single,numbers=left]
% \begin{tabular}{|l|ccc|}
% \hline
% \diagbox{Time}{Room}{Day} & Mon & Tue & Wed \\
% \hline
% Morning   & used & used &      \\
% Afternoon &      & used & used \\
% \hline
% \end{tabular}
% \end{Example}
%
% \subsection{更多参数设置}
%
% "\diagbox" 还可以在前面带一个可选参数,里面用 key-value 的语法设置宽度、方向
% 等更多的选项:
% \begin{description}
% \item[width] 明确指定盒子的总宽度。如果省略,则会自动计算能够放下所有内容的
% 宽度。
% \item[height] 明确指定盒子的总高度。
% \item[dir] 指定斜线方向。可以取 "NW"(西北)、"NE"(东北)、"SW"(西南)、
% "SE"(东南)四种方向。在只有一条斜线的表头中,"NE" 与 "SW"、 "SE" 与 "NW" 是
% 等价的。斜线方向的默认值是 "NW"。
%
% "\diagbox[dir="\meta{方向}"]{A}{B}" 在表格中的效果:\\
% \begin{tabular}{ll@{\qquad}ll}
% "NW" 或 "SE":&
% \begin{tabular}{|c|} \hline \diagbox[dir=SE]{A}{B} \\\hline \end{tabular} &
% "SW" 或 "NE":&
% \begin{tabular}{|c|} \hline \diagbox[dir=NE]{A}{B} \\\hline \end{tabular}
% \end{tabular}
%
% "\diagbox[dir="\meta{方向}"]{A}{M}{B}" 在表格中的效果:\\
% \begin{tabular}{*4{ll@{\qquad}}}
% "NW"&
% \begin{tabular}{|c|} \hline \diagbox[dir=NW]{A}{M}{B} \\\hline \end{tabular}&
% "NE"&
% \begin{tabular}{|c|} \hline \diagbox[dir=NE]{A}{M}{B} \\\hline \end{tabular}&
% "SW"&
% \begin{tabular}{|c|} \hline \diagbox[dir=SW]{A}{M}{B} \\\hline \end{tabular}&
% "SE"&
% \begin{tabular}{|c|} \hline \diagbox[dir=SE]{A}{M}{B} \\\hline \end{tabular}
% \end{tabular}
%
% \item[trim] 设置左边界或右边界不计算额外的空白,可以取值为 "l", "r", "lr" 或
% "rl"。这个选项在列格式包含 "@{}" 时将会有用。
% \end{description}
%
% \changes{v2.0}{2011/11/23}{变更 "trim" 选项的行为,去掉了使用 "trim" 选项时
% 内部的间距。这与 \pkg{slashbox} 行为不同。}
% 一个更复杂的例子:
% \begin{Example}[frame=single,numbers=left]
% \begin{tabular}{|@{}l|c|c|r@{}|}
% \hline
% \diagbox[width=5em,trim=l]{Time}{Day} & Mon & Tue & Wed\\
% \hline
% Morning   & used & used & used\\
% \hline
% Afternoon &      & used & \diagbox[dir=SW,height=2em,trim=r]{A}{B} \\
% \hline
% \end{tabular}
% \end{Example}
% \bigskip
%
% 此外,"\diagbox" 的表头内容还可以用 "\\" 手工换行。此时通常需要对自动计算的
% 表头高度进行手工调整。例如:
% \begin{Example}[frame=single,numbers=left]
% \begin{tabular}{|c|}
% \hline
% \diagbox[height=45pt]{line\\heads}{column\\heads} \\
% \hline
% \end{tabular}
% \end{Example}
%
% \subsection{对 \pkg{slashbox} 宏包的兼容性}
%
% 在使用 \pkg{diagbox} 宏包时,会模拟 \pkg{slashbox} 宏包的功能,并禁止
% \pkg{slashbox} 再被调用。
%
% \pkg{diagbox} 宏包提供了与 \pkg{slashbox} 大致相同的 "\slashbox" 与
% "\backslashbox" 两个命令。 "\slashbox" 与 "\backslashbox" 的语法来自
% \pkg{slashbox} 宏包,排版效果略有区别。这两个命令仅在旧文档中作为兼容命令使
% 用。实际中使用 "\diagbox" 更为方便。
%
% \DescribeMacro{\backslashbox}
% "\backslashbox" 基本功能与 "\diagbox" 类似。它带有两个可选参数,分别表示
% "\diagbox" 中的 "width" 与 "trim" 选项。
%
% \DescribeMacro{\slashbox}
% "\slashbox" 基本功能与 "\diagbox[dir=SW]" 类似。它也带有两个可选参数,表示
% "\diagbox" 中的 "width" 和 "tirm" 选项。
%
% 例如:\\
% \begin{SideBySideExample}[frame=single,numbers=left,xrightmargin=.45\linewidth]
% \begin{tabular}{|c|c|c|} \hline
% \backslashbox[2cm]{num}{alpha}
%   & A  & B  \\\hline
% 1 & A1 & B1 \\\hline
% 2 & A2 & B2 \\\hline
% \end{tabular}
% \end{SideBySideExample}
%
% \section{已知问题和未来版本}
%
% 已知问题:
% \begin{itemize}
% \item "\slashbox" 与 "\backslashbox" 命令的效果与在 \pkg{slashbox} 宏包中不
% 同。两个宏包在计算盒子宽度和高度时,使用了不同的算法;同时,在处理
% "\slashbox" 第二个可选参数(即 "\diagbox" 的 "trim" 键)时,使用的方式也不一
% 样。
%
% 这不是 bug。通常 \pkg{diagbox} 计算出的宽度比 \pkg{slashbox} 的结果更安全一
% 些。
%
% \item
% "\diagbox" 生成的单元格必须是表列中最宽的一个。如果不能达到最宽,则画出的斜
% 线不能保证在正确的位置。例如:\\[1ex]
% \begin{SideBySideExample}[frame=single,numbers=left,xrightmargin=.45\linewidth]
% \begin{tabular}{|c|}  \hline
%   \diagbox{A}{B}    \\\hline
%   Very long term    \\\hline
% \end{tabular}
% \end{SideBySideExample}
% \\[1ex]
% 此时可以手工设置较宽的 "\diagbox" 的 "width" 选项,解决此问题。
% \end{itemize}
%
% 未尽的工作:
% \begin{itemize}
% \item 源代码的文档需要改进。特别是在 "\diagbox@triple" 中的宽度和高度计算算
% 法需要详细说明。
% \end{itemize}
%
% \StopEventually{}
%
%
% \clearpage
% \section{Implementation / 代码实现}
%
% \iffalse
%<*package>
% \fi
%
%
% 使用 key-value 界面。
%    \begin{macrocode}
\RequirePackage{keyval}
%    \end{macrocode}
% 绘图依赖 \pkg{pict2e} 宏包。
%    \begin{macrocode}
\RequirePackage{pict2e}
%    \end{macrocode}
% 计算依赖 \pkg{fp} 宏包。
%    \begin{macrocode}
\RequirePackage[nomessages]{fp}
%    \end{macrocode}
% 
% 分配用到的盒子寄存器。它们分别对应于 "\diagbox" 三个必选参数的内容。
%    \begin{macrocode}
\newbox\diagbox@boxa
\newbox\diagbox@boxb
\newbox\diagbox@boxm
%    \end{macrocode}
% 分配长度变量。
%    \begin{macrocode}
\newdimen\diagbox@wd
\newdimen\diagbox@ht
\newdimen\diagbox@sepl
\newdimen\diagbox@sepr
%    \end{macrocode}
% 
% 定义 "\diagbox" 的键值选项。
%    \begin{macrocode}
\define@key{diagbox}{width}{%
  \setlength{\diagbox@wd}{#1}}
\define@key{diagbox}{height}{%
  \setlength{\diagbox@ht}{#1}}
\define@key{diagbox}{trim}{%
  \@tfor\@reserveda:=#1\do{%
    \ifcsname diagbox@sep\@reserveda\endcsname
      \setlength{\csname diagbox@sep\@reserveda\endcsname}{\z@}%
    \else
      \PackageError{diagbox}{Unknown trim option `#1'.}{l, r, lr and rl are supported.}%
    \fi}}
\define@key{diagbox}{dir}{%
  \def\diagbox@dir{#1}%
  \unless\ifcsname diagbox@dir@#1\endcsname
    \PackageError{diagbox}{Unknown direction `#1'.}{NW, NE, SW, SE are supported.}%
    \def\diagbox@dia{NW}%
  \fi}
\let\diagbox@dir@SE\relax
\let\diagbox@dir@SW\relax
\let\diagbox@dir@NE\relax
\let\diagbox@dir@NW\relax
%    \end{macrocode}
%
% \begin{macro}{\diagbox@pict}
% 这是带斜线的盒子本身。由一个 "picture" 环境实现。
%    \begin{macrocode}
\def\diagbox@pict{%
  \unitlength\p@
  \begin{picture}
    (\strip@pt\dimexpr\diagbox@wd-\diagbox@sepl-\diagbox@sepr\relax,\strip@pt\diagbox@ht)
    (\strip@pt\diagbox@sepl,0)
      \@nameuse{diagbox@\diagbox@part @pict@\diagbox@dir}
  \end{picture}}
%    \end{macrocode}
% \end{macro}
%
%
% \begin{macro}{\diagbox@double@pict@SE}
% 方向为 "SE" 的斜线盒子内容。
%    \begin{macrocode}
\def\diagbox@double@pict@SE{%
  \put(0,0) {\makebox(0,0)[bl]{\box\diagbox@boxa}}
  \put(\strip@pt\diagbox@wd,\strip@pt\diagbox@ht) {\makebox(0,0)[tr]{\box\diagbox@boxb}}
  \Line(0,\strip@pt\diagbox@ht)(\strip@pt\diagbox@wd,0)}
%    \end{macrocode}
% \end{macro}
%
% \begin{macro}{\diagbox@double@pict@NW}
% 方向 "NW" 与 "SE" 相同。
%    \begin{macrocode}
\let\diagbox@double@pict@NW\diagbox@double@pict@SE
%    \end{macrocode}
% \end{macro}
%
% \begin{macro}{\diagbox@double@pict@NE}
% 方向为 "NE" 的斜线盒子内容。
%    \begin{macrocode}
\def\diagbox@double@pict@NE{%
  \put(0,\strip@pt\diagbox@ht) {\makebox(0,0)[tl]{\box\diagbox@boxa}}
  \put(\strip@pt\diagbox@wd,0) {\makebox(0,0)[br]{\box\diagbox@boxb}}
  \Line(0,0)(\strip@pt\diagbox@wd,\strip@pt\diagbox@ht)}
%    \end{macrocode}
% \end{macro}
%
% \begin{macro}{\diagbox@double@pict@NE}
% 方向 "SW" 与 "NE" 相同。
%    \begin{macrocode}
\let\diagbox@double@pict@SW\diagbox@double@pict@NE
%    \end{macrocode}
% \end{macro}
%
%
% \begin{macro}{\diagbox@double}
% \changes{v2.0}{2011/11/22}{在使用 "trim" 选项时去掉内容与盒子边界的间距。这
% 与 \pkg{slashbox} 的行为不同。}
% 分成两部分的盒子。三个参数,分别为 key-value 格式的可选项、左半边内容、右半边
% 内容。这里的主要工作是读入参数并计算斜线盒子的大小。
%    \begin{macrocode}
\def\diagbox@double#1#2#3{%
  \begingroup
  \diagbox@wd=\z@
  \diagbox@ht=\z@
  \diagbox@sepl=\tabcolsep
  \diagbox@sepr=\tabcolsep
  \def\diagbox@part{double}%
  \setkeys{diagbox}{dir=NW,#1}%
  \setbox\diagbox@boxa=\hbox{%
    \begin{tabular}{@{\hspace{\diagbox@sepl}}l@{}}#2\end{tabular}}%
  \setbox\diagbox@boxb=\hbox{%
    \begin{tabular}{@{}r@{\hspace{\diagbox@sepr}}}#3\end{tabular}}%
  \ifdim\diagbox@wd=\z@
    \ifdim\wd\diagbox@boxa>\wd\diagbox@boxb
      \diagbox@wd=\dimexpr2\wd\diagbox@boxa+\diagbox@sepl+\diagbox@sepr\relax
    \else
      \diagbox@wd=\dimexpr2\wd\diagbox@boxb+\diagbox@sepl+\diagbox@sepr\relax
    \fi
  \fi
  \ifdim\diagbox@ht=\z@
    \diagbox@ht=\dimexpr\ht\diagbox@boxa+\dp\diagbox@boxa+\ht\diagbox@boxb+\dp\diagbox@boxb\relax
  \fi
  $\vcenter{\hbox{\diagbox@pict}}$%
  \endgroup}
%    \end{macrocode}
% \end{macro}
%
% \begin{macro}{\diagbox@triple@setbox@NW}
%    \begin{macrocode}
\def\diagbox@triple@setbox@NW#1#2#3{%
  \setbox\diagbox@boxa=\hbox{%
    \begin{tabular}{@{\hspace{\diagbox@sepl}}l@{}}#1\end{tabular}}%
  \setbox\diagbox@boxm=\hbox{%
    \begin{tabular}{@{\hspace{\diagbox@sepl}}l@{}}#2\end{tabular}}%
  \setbox\diagbox@boxb=\hbox{%
    \begin{tabular}{@{}r@{\hspace{\diagbox@sepr}}}#3\end{tabular}}}
%    \end{macrocode}
% \end{macro}
%
% \begin{macro}{\diagbox@triple@setbox@SW}
%    \begin{macrocode}
\let\diagbox@triple@setbox@SW\diagbox@triple@setbox@NW
%    \end{macrocode}
% \end{macro}
%
% \begin{macro}{\diagbox@triple@setbox@NW}
%    \begin{macrocode}
\def\diagbox@triple@setbox@SE#1#2#3{%
  \setbox\diagbox@boxa=\hbox{%
    \begin{tabular}{@{\hspace{\diagbox@sepl}}l@{}}#1\end{tabular}}%
  \setbox\diagbox@boxm=\hbox{%
    \begin{tabular}{@{}r@{\hspace{\diagbox@sepr}}}#2\end{tabular}}%
  \setbox\diagbox@boxb=\hbox{%
    \begin{tabular}{@{}r@{\hspace{\diagbox@sepr}}}#3\end{tabular}}}
%    \end{macrocode}
% \end{macro}
%
% \begin{macro}{\diagbox@triple@setbox@NE}
%    \begin{macrocode}
\let\diagbox@triple@setbox@NE\diagbox@triple@setbox@SE
%    \end{macrocode}
% \end{macro}
%
%
% \begin{macro}{\diagbox@triple@pict@NW}
%    \begin{macrocode}
\def\diagbox@triple@pict@NW{%
  \put(0,0)   {\makebox(0,0)[bl]{\box\diagbox@boxa}}
  \put(0,\y)  {\makebox(0,0)[tl]{\box\diagbox@boxm}}
  \put(\x,\y) {\makebox(0,0)[tr]{\box\diagbox@boxb}}
  \Line(0,\yym)(\x,0)
  \Line(\xm,\y)(\x,0)}
%    \end{macrocode}
% \end{macro}
%
%
% \begin{macro}{\diagbox@triple@pict@NE}
%    \begin{macrocode}
\def\diagbox@triple@pict@NE{%
  \put(0,\y)  {\makebox(0,0)[tl]{\box\diagbox@boxa}}
  \put(\x,\y) {\makebox(0,0)[tr]{\box\diagbox@boxm}}
  \put(\x,0)  {\makebox(0,0)[br]{\box\diagbox@boxb}}
  \Line(0,0)(\xxm,\y)
  \Line(0,0)(\x,\yym)}
%    \end{macrocode}
% \end{macro}
%
%
% \begin{macro}{\diagbox@triple@pict@SW}
%    \begin{macrocode}
\def\diagbox@triple@pict@SW{%
  \put(0,\y) {\makebox(0,0)[tl]{\box\diagbox@boxa}}
  \put(0,0)  {\makebox(0,0)[bl]{\box\diagbox@boxm}}
  \put(\x,0) {\makebox(0,0)[br]{\box\diagbox@boxb}}
  \Line(0,\ym)(\x,\y)
  \Line(\xm,0)(\x,\y)}
%    \end{macrocode}
% \end{macro}
%
%
% \begin{macro}{\diagbox@triple@pict@SE}
%    \begin{macrocode}
\def\diagbox@triple@pict@SE{%
  \put(0,0)   {\makebox(0,0)[bl]{\box\diagbox@boxa}}
  \put(\x,0)  {\makebox(0,0)[br]{\box\diagbox@boxm}}
  \put(\x,\y) {\makebox(0,0)[tr]{\box\diagbox@boxb}}
  \Line(0,\y)(\xxm,0)
  \Line(0,\y)(\x,\ym)}
%    \end{macrocode}
% \end{macro}
%
%
% \begin{macro}{\diagbox@triplebox}
% \changes{v2.0}{2011/11/22}{新增三部分双斜线的盒子}
% 分成三部分的盒子。四个参数,分别为 key-value 格式的可选项、左半边内容、中间
% 内容、右半边内容。
%    \begin{macrocode}
\def\diagbox@triple#1#2#3#4{%
  \begingroup
  \diagbox@wd=\z@
  \diagbox@ht=\z@
  \diagbox@sepl=\tabcolsep
  \diagbox@sepr=\tabcolsep
  \def\diagbox@part{triple}%
  \setkeys{diagbox}{dir=NW,#1}%
  \@nameuse{diagbox@triple@setbox@\diagbox@dir}{#2}{#3}{#4}%
%    \end{macrocode}
% 取长宽
%    \begin{macrocode}
  \edef\xa{\strip@pt\wd\diagbox@boxa}%
  \edef\ya{\strip@pt\dimexpr\ht\diagbox@boxa+\dp\diagbox@boxa\relax}%
  \edef\xb{\strip@pt\wd\diagbox@boxb}%
  \edef\yb{\strip@pt\dimexpr\ht\diagbox@boxb+\dp\diagbox@boxb\relax}%
  \edef\xm{\strip@pt\wd\diagbox@boxm}%
  \edef\ym{\strip@pt\dimexpr\ht\diagbox@boxm+\dp\diagbox@boxm\relax}%
%    \end{macrocode}
% 列方程,求方程系数
%    \begin{macrocode}
  \FPneg\bi\yb
  \FPadd\ci\xb\xm  \FPneg\ci\ci
  \FPmul\di\xm\yb
  \FPadd\bj\ya\ym  \FPneg\bj\bj
  \FPneg\cj\xa
  \FPmul\dj\xa\ym
%    \end{macrocode}
% 解方程
%    \begin{macrocode}
  \FPsub\u\dj\di
  \FPupn{v}{bj ci * bi cj * -}%
  \FPupn{delta}{bi dj * bj di * - cj ci - * 4 * %
    v u + copy * %
    - 2 swap root}%
  \ifdim\diagbox@wd=\z@
    \FPupn{x}{2 bj bi - delta v u - + / /}%
    \diagbox@wd=\x\p@
  \else
    \edef\x{\strip@pt\diagbox@wd}%
  \fi
  \ifdim\diagbox@ht=\z@
    \FPupn{y}{2 cj ci - delta v u + - / /}%
    \diagbox@ht=\y\p@
  \else
    \edef\y{\strip@pt\diagbox@ht}%
  \fi
  \FPsub\xxm\x\xm
  \FPsub\yym\y\ym
%    \end{macrocode}
% 画盒子
%    \begin{macrocode}
  $\vcenter{\hbox{\diagbox@pict}}$%
  \endgroup}
%    \end{macrocode}
% \end{macro}
%
%
% \begin{macro}{\diagbox}
% \changes{v2.0}{2011/11/22}{判断参数个数,选择两部分或三部分盒子。}
% 主要的用户命令。判断使用两部分还是三部分的盒子。
%    \begin{macrocode}
\newcommand\diagbox[3][]{%
  \@ifnextchar\bgroup
    {\diagbox@triple{#1}{#2}{#3}}{\diagbox@double{#1}{#2}{#3}}}
%    \end{macrocode}
% \end{macro}
%
% 以下代码用来模拟 \pkg{slashbox} 宏包的功能。
%
% 禁止读入 \pkg{slashbox}。
%    \begin{macrocode}
\expandafter\xdef\csname ver@slashbox.\@pkgextension\endcsname{9999/99/99}
%    \end{macrocode}
%
%
% \begin{macro}{\slashbox}
% 模拟 "\slashbox"。
%    \begin{macrocode}
\def\slashbox{%
  \def\diagbox@slashbox@options{dir=SW,}%
  \slashbox@}
%    \end{macrocode}
% \end{macro}
%
%
% \begin{macro}{\backslashbox}
% 模拟 "\backslashbox"。
%    \begin{macrocode}
\def\backslashbox{%
  \def\diagbox@slashbox@options{dir=NW,}%
  \slashbox@}
%    \end{macrocode}
% \end{macro}
%
%
% \begin{macro}{\slashbox@}
%    \begin{macrocode}
\newcommand\slashbox@[1][]{%
  \ifx\relax#1\relax\else
    \edef\diagbox@slashbox@options{%
      \unexpanded\expandafter{\diagbox@slashbox@options}%
      \unexpanded{width=#1,}}%
  \fi
  \slashbox@@}
%    \end{macrocode}
% \end{macro}
%
%
% \begin{macro}{\slashbox@@}
%    \begin{macrocode}
\newcommand\slashbox@@[3][]{%
  \edef\diagbox@slashbox@options{%
    \unexpanded\expandafter{\diagbox@slashbox@options}%
    \unexpanded{trim=#1,}}%
  \expandafter\diagbox\expandafter[\diagbox@slashbox@options]{#2}{#3}}
\endinput
%    \end{macrocode}
% \end{macro}
%
%
% \iffalse
%</package>
% \fi
%
% \Finale
\endinput
